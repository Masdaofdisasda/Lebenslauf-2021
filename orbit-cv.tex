\documentclass[9pt]{orbit-cv}
\urlstyle{rm}
\begin{document}


%%%%%%%%%%%%%%%
% Color schemes
%%%%%%%%%%%%%%%%

% Uncomment one of these if you'd rather
% not use the default color scheme

%\colorschemeTwo
%\colorschemeThree
%\colorschemeFour
%\colorschemeFive
\colorschemeSix

\pagecolor{background}
%%%%%%%%%
% Profile
%%%%%%%%%

%\leftSidebar{}
\cvname{David Köppl} %your name
\cvjobtitle{Student}%your actual job position
\profilepic{profile.jpg} %path of profile pic

\addContact{\faEnvelope{}}{\href{mailto:david.koeppl75@gmail.com}{david.koeppl75@gmail.com}}
\addContact{\faPhone{}}{+43 650 5241895}
\addContact{\faGlobe{}}{\href{http://davidkoeppl.com}{www.davidkoeppl.com}}
\addContact{\faGithub{}}{\href{http://github.com/masdaofdisasda}{GitHub}}
%\addContact{\faExternalLink{}}{\href{http://rusty.example.com/pdf/cv-rusty.pdf}{ausführlicher Lebenslauf}}

%% Bildung
\addEducation{B.Sc. in Medieninformatik und Visual Computing}{TU Wien}{2020-heute}
\addEducation{Matura}{HTL Mödling - Höhere Abteilung für Elektrotechnik}{2013-2018}
\addEducation{Unterstufe}{BG/BRG Keimgasse - Informatikzweig}{2009-2013}
%% Sprachen
\addLanguage{Deutsch}{Muttersprache}
\addLanguage{Englisch}{professionell}
%% Hobbies
\addInterest{\faMusic{}  Musikproduktion}
\addInterest{\faCode{} Coding}
\addInterest{\faHeartbeat{} Calisthenics}


%% (mad!)Skillzz
\addSkill{\LaTeX}{0.65}
\addSkill{Java}{0.85}
\addSkill{Gardening}{0.92}
\addSkill{Standing Still}{0.99}
\addSkill{Conversation}{0.40}
\addSkill{Sculpting}{0.80}

\makeprofile


\section{\faUser{} QUALIFIKATIONEN}
    \begin{itemize}
     \item Erfahrung mit objektorientierter Programmierung und Web-Design
     \item flexible Anpassung an Situationen und zielorientiertes Time-Management
     \item lernwilliges und effektives Problemlösungsdenken
    \end{itemize}

\section{\faBriefcase{} ERFAHRUNG}
\workexp{Lichtplaner/Technischer Innendienst}{2020}{Lichtprojekt Aigner \& Wöber GmbH}{
  
    \begin{itemize}
     \item Entwurf, Planung und Umsetzung einer neuen LED-Profilleuchte
     \item genaue Lichtberechnung für mehrstöckige Bauprojekte (z.B. Heidi Horten Museum)
     \item effektivere Artikelnummern-Vergabe durch Schreiben eines Programms in VBA
    \end{itemize}
}

\workexp{Rettungssanitäter (FSJ)}{2019}{Rotes Kreuz Mödling}{

    \begin{itemize}
     \item angehende Sanitäter in ihrer Ausbildung begleitet und geschult
     \item respektvolle Zusammenarbeit mit vielfältigen Kollegen und Einsatz-Teams
     \item Evaluierung und Erfassung von Patienten und deren Verletzungen
    \end{itemize}
}

\workexp{Elektriker (Praktikum)}{2017}{Elektro Hartmann}{

    \begin{itemize}
     \item Zusammenbau, Installation und Wartung elektrischer Komponenten 
     \item Fehlersuche und normgerechte Behebung unter Zeitdruck
     \item problemlos eigenständige, sowie Team-basierte Arbeit geleistet
    \end{itemize}
}

\workexp{Feinkost-Mitarbeiter (geringfügig)}{2013 — 2018}{Billa AG}{

    \begin{itemize}
     \item selbständige Arbeitsplanung um die Vollständigkeit des Sortiments zu garantieren
     \item regelmäßige Inventur der Ware sowie Abschreiben von alten Produkten
     \item freundlicher und hilfsbereiter Kundenumgang 
    \end{itemize}
}

\section{\faArchive{} AUSGEWÄHLTE PROJEKTE}


\project{https://github.com/Masdaofdisasda/Barnes-Hut_Universe_Simulation}{Barnes-Hut Weltraum Simulation}{
  Ziel diese Projekts war es eine Datenstruktur in Java zu entwerfen, welche die Anziehungskraft von mindestens 10.000 verschiedene Himmelskörpern mit unterschiedlichen Eigenschaften wie Sterne, Planeten und schwarze Löcher, in Echtzeit berechnen kann. Dafür habe ich einen Octree geschrieben der den Barnes-Hut Algorithmus implementiert, wodurch immer nur Cluster von Planeten betrachtet werden, was die Berechnung in Echtzeit problemlos ermöglichen könnte.
}

\project{https://www.davidkoeppl.com}{Portfolio Webseite}{
  Um mir selbst Wordpress und CSS beizubringen, habe ich diese Webseite aufgebaut. Sie dient als Portfolio für meine Coding-Projekte und mein Musik-Projekt. Das Design sowie alle Fotos wurden von mir eigens erstellt und bearbeitet.
}

\section{\faRocket{} FÄHIGKEITEN \& FERTIGKEITEN}{
Assembler: Atmel, Micro16

Kommandozeile: PowerShell, zsh

Objektorientiert: C\#, Java

Scripting: LUA, Python, Windows Shell

Webentwicklung: HTML, CSS, Wordpress

Akademisch: MATLAB, LaTex

SPS: FUP, KOP

3D Modelling: Blender, AutoCAD, Fusion 360, DiaLUX

Design: Photoshop, Lightroom, Illustrator
}

%\ListSkills{}
%\input{bars}


\end{document} 
%%% Local Variables:
%%% mode: latex
%%% TeX-master: t
%%% End:
